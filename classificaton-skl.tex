\huge
\[ \mbox{ Classification with Scikit-Learn} \]

  

\textbf{Understanding Classification}\\
Although regression and classification appear to be very different they are in fact similar problems.

<ul>
 In regression our predictions for the response are real-valued numbers
 on the other hand, in classification the response is a mutually exclusive class label 
 Example ``\textit{Is the email spam?}" or ``\textit{Is the credit card transaction fraudulent?}".
<\ul>


  


 
 \textbf{Binary Classsification Problems}\\
 <ul>
 If the number of classes is equal to two, then we call it a binary classification problem; if there are more than two classes, then we call it a multiclass classification problem.
  In the following we will assume binary classification because it’s the more general case, and — we can always represent a multiclass problem as a sequence of binary classification problems.
<\ul>


  


\textbf{Credit Card Fraud}
<ul>
 We can also think of classification as a function estimation problem where the function that we want to estimate separates the two classes. 
 This is illustrated in the example below where our goal is to predict whether or not a credit card transaction is fraudulent
 he dataset is provided by James et al., \textbf{Introduction to Statistical Learning}.
<\ul>



  


\vspace{-1cm}
\textbf{Credit Card Fraud}
\begin{figure}
\centering
\includegraphics[width=1.2\linewidth]{sklcass/sklclass1}

\end{figure}


  


\textbf{Credit Card Fraud}
\begin{figure}
\centering
\includegraphics[width=0.7\linewidth]{sklcass/sklclass2}

\end{figure}



  


\textbf{Credit Card Fraud}
<ul>
 On the left you can see a scatter plot where fraudulent cases are red dots and non-fraudulent cases are blue dots. 
 A good separation seems to be a vertical line at around a balance of 1400 as indicated by the boxplots on the next slide.
<\ul>




  


\begin{figure}
\centering
\includegraphics[width=0.95\linewidth]{sklcass/sklclass3}

\end{figure}




 #===
\begin{figure}
\centering
\includegraphics[width=0.9\linewidth]{sklcass/sklclass4}
\end{figure}
 #===
